\documentclass[twocolumn]{article}
 
\usepackage{NLP}
\usepackage[dvipdfmx]{graphicx}
\usepackage{fancybox}

\title{\textbf{Multi-Language Sentiment Analysis of SNS across Different Cities}}

\author{
\begin{tabular}{cc}
~~~~~Muga Yoshikawa & Hiroaki Saito \\ 
\multicolumn{2}{c}{Keio Universiry Graduate School of Science and Technology} \\ 
\vspace{-4ex}
\end{tabular}} 

\date{\texttt{\{muga, hxs\}@nak.ics.keio.ac.jp}}

\begin{document}
\maketitle

\section{Introduction}
\vspace{-2mm}
Sentiment Analysis is a classification of sentences by labelling. 
Given a sentence, it distinguishes if the input means something happy, negative or neutral.
These days it is applied to many areas.
For example, movie or product reviews \cite{movie_review} and opinion surveys for elections \cite{us_election}.
Sentiment analysis of Social Network Service (SNS) especially has potential to see what people in the Internet are thinking of, what images they are having on some topics.
Since nowadays billions of people in the world are using smartphones, sentiment analysis of SNS can be helpful to see opinions for certain topic (e.g. TV show, concerts etc.). 

As the EU referendum in U.K in June 2016 shows, the result varied obviously by areas in the country \cite{uk_referendum}.
Not only this vote, this phenomena are observed also in U.S. presidential election in November 2016 \cite{us_map}.
Thus, sentimental analysis for different location is necessary task of natural language processing.

In addition, we focus on not only one language but also other languages because even people in the same country or city speak different languages, which happens usually in many places these days.
With more languages, more precise data can be obtained.
According to a psychology researches \cite{psychology1} \cite{psychology2}, if people speak in different languages they do not think in the same way.
Only the perspective of one language is not enough for observing trends especially in big cities where many people from different countries around the world gather.

In this research we propose the system of multi-language sentiment analysis of SNS, specifically Twitter, and we analyse across different locations in different countries.
With public available twitter data that are annotated by human, we build classifiers to label tweet obtained by locations.
We picked around 20 cities where English, French, German or Spanish is spoken as official language and the results of every classifier shows the validity of this system (?).

\vspace{-6mm}

\section{Related Work}
\vspace{-2mm}
related work 1, 2, 3
\vspace{-6mm}

\section{Twitter API}
\vspace{-2mm}
explanation of twitter place ID
\vspace{-6mm}

\section{Experiment}
\vspace{-2mm}
experiment
\vspace{-6mm}

\section{Result}
\vspace{-2mm}
result
\vspace{-6mm}

\section{Conclusion}
\vspace{-2mm}
Conclusion
\vspace{-6mm}
\bibliographystyle{plain}
\bibliography{./bibliography}
\end{document}
